% Arquivo LaTeX de exemplo de monografia para a disciplina MAC0499
%
% Adaptado em julho/2015 a partir do
%
% ---------------------------------------------------------------------------- %
% Arquivo LaTeX de exemplo de dissertação/tese a ser apresentados à CPG do IME-USP
%
% Versão 5: Sex Mar  9 18:05:40 BRT 2012
%
% Criação: Jesús P. Mena-Chalco
% Revisão: Fabio Kon e Paulo Feofiloff


\documentclass[12pt,twoside,a4paper,openany]{book}


% ---------------------------------------------------------------------------- %
% Pacotes
\usepackage[T1]{fontenc}
\usepackage[brazil]{babel}
\usepackage[utf8]{inputenc}
\usepackage[pdftex]{graphicx}           % usamos arquivos pdf/png como figuras
\usepackage{setspace}                   % espaçamento flexível
\usepackage{indentfirst}                % indentação do primeiro parágrafo
\usepackage{makeidx}                    % índice remissivo
\usepackage[nottoc]{tocbibind}          % acrescentamos a bibliografia/indice/conteudo no Table of Contents
\usepackage{courier}                    % usa o Adobe Courier no lugar de Computer Modern Typewriter
\usepackage{type1cm}                    % fontes realmente escaláveis
\usepackage{listings}                   % para formatar código-fonte (ex. em Java)
%\usepackage{titletoc}
\usepackage{minitoc}
%\usepackage[bf,small,compact]{titlesec} % cabeçalhos dos títulos: menores e compactos
%\usepackage[fixlanguage]{babelbib}
\usepackage{url}
\usepackage[font=small,format=plain,labelfont=bf,up,textfont=it,up]{caption}
\usepackage[usenames,svgnames,dvipsnames]{xcolor, colortbl}
\usepackage[a4paper,top=2.54cm,bottom=2.0cm,left=2.0cm,right=2.54cm]{geometry} % margens
%\usepackage[pdftex,plainpages=false,pdfpagelabels,pagebackref,colorlinks=true,citecolor=black,linkcolor=black,urlcolor=black,filecolor=black,bookmarksopen=true]{hyperref} % links em preto
\usepackage[pdftex,plainpages=false,pdfpagelabels,pagebackref,colorlinks=true,citecolor=DarkGreen,linkcolor=NavyBlue,urlcolor=DarkRed,filecolor=green,bookmarksopen=true]{hyperref} % links coloridos
\usepackage[all]{hypcap}                    % soluciona o problema com o hyperref e capitulos
\usepackage[round,sort,nonamebreak]{natbib} % citação bibliográfica textual(plainnat-ime.bst)
\usepackage{emptypage}  % para não colocar número de página em página vazia
\usepackage{subfig}                     %sub-imagens
\usepackage{bookmark}
\usepackage{dirtytalk}


\fontsize{60}{62}\usefont{OT1}{cmr}{m}{n}{\selectfont}

% ---------------------------------------------------------------------------- %
% Cabeçalhos similares ao TAOCP de Donald E. Knuth
\usepackage{fancyhdr}
\pagestyle{fancy}
\fancyhf{}
\renewcommand{\chaptermark}[1]{\markboth{\MakeUppercase{#1}}{}}
\renewcommand{\sectionmark}[1]{\markright{\MakeUppercase{#1}}{}}
\renewcommand{\headrulewidth}{0pt}
\addto{\captionsbrazil}{% Making babel aware of special titles
\renewcommand{\mtctitle}{Neste capítulo}}

% ---------------------------------------------------------------------------- %
\graphicspath{{./figuras/}}             % caminho das figuras (recomendável)
\frenchspacing                          % arruma o espaço: id est (i.e.) e exempli gratia (e.g.)
\urlstyle{same}                         % URL com o mesmo estilo do texto e não mono-spaced
\makeindex                              % para o índice remissivo
\raggedbottom                           % para não permitir espaços extra no texto
\fontsize{60}{62}\usefont{OT1}{cmr}{m}{n}{\selectfont}
\cleardoublepage
\normalsize

% ---------------------------------------------------------------------------- %
% Opções de listing usados para o código fonte
% Ref: http://en.wikibooks.org/wiki/LaTeX/Packages/Listings
\lstset{ %
language=Ruby,                  % choose the language of the code
basicstyle=\footnotesize,       % the size of the fonts that are used for the code
numbers=left,                   % where to put the line-numbers
numberstyle=\footnotesize,      % the size of the fonts that are used for the line-numbers
stepnumber=1,                   % the step between two line-numbers. If it's 1 each line will be numbered
numbersep=5pt,                  % how far the line-numbers are from the code
showspaces=false,               % show spaces adding particular underscores
showstringspaces=false,         % underline spaces within strings
showtabs=false,                 % show tabs within strings adding particular underscores
frame=single,	                % adds a frame around the code
framerule=0.6pt,
tabsize=2,	                    % sets default tabsize to 2 spaces
captionpos=b,                   % sets the caption-position to bottom
breaklines=true,                % sets automatic line breaking
breakatwhitespace=false,        % sets if automatic breaks should only happen at whitespace
escapeinside={\%*}{*)},         % if you want to add a comment within your code
backgroundcolor=\color[rgb]{1.0,1.0,1.0}, % choose the background color.
rulecolor=\color[rgb]{0.8,0.8,0.8},
extendedchars=true,
xleftmargin=10pt,
xrightmargin=10pt,
framexleftmargin=10pt,
framexrightmargin=10pt
}
\newcommand{\codigo}[5]{\lstinputlisting[firstline={#1},lastline={#2}, caption={#3}, label={#4}]{#5}}
\newcommand{\paragrafo}[1]{\textbf{\texttt{{#1}}}}
\newcommand{\kw}[1]{\textbf{{#1}}}
\newcommand{\dir}{$\rightarrow$ }
\newcommand{\intro}[1]{\vfill
\begin{center}
{\LARGE \textbf{--- Introdução ---}}
\end{center}
{#1} \newpage}
\newcommand{\Oh}[1]{$O({#1})$}

\defcitealias{MN2}{manual do NEHiLP}
\defcitealias{Rcook:09}{Ruby Cookbook}
\defcitealias{RoR}{Ruby on Rails}
\defcitealias{Nuxt}{Nuxt.js}
\defcitealias{Vue}{Vue.js}
\defcitealias{Jest}{Jest}

% ---------------------------------------------------------------------------- %
% Corpo do texto
\begin{document}

    \frontmatter
    % cabeçalho para as páginas das seções anteriores ao capítulo 1 (frontmatter)
    \fancyhead[RO]{{\footnotesize\rightmark}\hspace{2em}\thepage}
    \setcounter{tocdepth}{2}
    \fancyhead[LE]{\thepage\hspace{2em}\footnotesize{\leftmark}}
    \fancyhead[RE,LO]{}
    \fancyhead[RO]{{\footnotesize\rightmark}\hspace{2em}\thepage}

    \onehalfspacing  % espaçamento


% ---------------------------------------------------------------------------- %
% CAPA
    \thispagestyle{empty}
    \begin{center}
        \vspace*{2.3cm}
        Universidade de São Paulo\\
        Instituto de Matemática e Estatística \\
        Bacharelado  em Ciência da Computação


        \vspace*{3cm}
        \Large{André Luiz Abdalla Silveira}


        \vspace{3cm}
        \textbf{\Large{DELPo \\
             -- Continuando a refatoração --}}


        \vskip 5cm
        \normalsize{São Paulo}

        \normalsize{15 de Novembro de 2.019}
    \end{center}
    % ---------------------------------------------------------------------------- %
    % Página de rosto
    %
    \newpage
    \thispagestyle{empty}
    \begin{center}
        \vspace*{2.3 cm}
        \textbf{\Large{DELPo \\
        -- Continuando a refatoração --}}
        \vspace*{2 cm}
    \end{center}

    \vskip 2cm

    \begin{flushright}
        Monografia final da disciplina \\
        MAC0499 -- Trabalho de Formatura Supervisionado.
    \end{flushright}

    \vskip 5cm

    \begin{center}
        Supervisor: Prof. Dr. Marco Dimas Gubitoso

        \vskip 5cm
        \normalsize{São Paulo}

        \normalsize{15 de Novembro de 2.019}
    \end{center}
    \pagebreak




    \pagenumbering{roman}     % começamos a numerar

    % ---------------------------------------------------------------------------- %
    % Agradecimentos:
    \chapter*{Agradecimentos}

    Meus agradecimentos pricipais são a Deus, meus adorados pais que me proporcianram
    (e continuam a fazê-lo) as melhores oportunidades que eu poderia ter. Meu
    agradecimento também dirige-se ao meu orientador que me apoiou em todos os momentos
    do meu trabalho. Agradeço também por ter me introduzido ao pessoal do NEHiLP que
    me proporcionou a graça de trabalhar com o DELPo. Tem sido uma experiencia e tanto.

    Outro agradecimento importante dirige-se ao pessoal da Codeminer 42, empresa em que
    trabalhei por algum tempo. Em especial a Elias e Alessandro. O primeiro trabalhava
    a meu lado e sempre me ajudava quando tinha disponibilidade. O último também deu
    uma grande ajuda, e além de tirar algumas dúvidas, forçava-me a escrever um código
    que seguisse, da melhor forma possível, os princípios do bom código


    % ---------------------------------------------------------------------------- %
    % Resumo
    \chapter*{Resumo}

    O projeto do Dicionário Etimológico da Língua Portuguesa (DELPo) nasceu da necessidade
    dos pesquisadores que se dedicam ao estudo da etimologia do idioma contarem com
    informações a respeito da origem e do significado dos verbetes. O dicionário estará
    disponível online, mas de forma incompleta, dado que muitas coisas foram refatoradas
    como o banco de dados (agora em PostgreSQL), módulos foram refatorados de forma que as
    dependências fiquem no seu devido lugar. A qualidade do código foi peça chave nessa
    refatoração, bem como a corbertura completa de testes e a implementação de uma interface
    agradável ao cliente (pesquisadores do NEHiLP)

    \noindent \textbf{Palavras-chave:} dicionário, etimologia, banco de dados,
    desenvolvimento web, testes, UI

    % ---------------------------------------------------------------------------- %
    % Abstract
    \chapter*{Abstract}


    The project `Dicionário Etimológico da Língua Portuguesa (DELPo)' came from the need
    of researchers dedicated to etymology studies to count on information regarding the origin
    and meaning of entries.
    The dictionary will be available online, in spite of being incomplete, given that there are
    many things that have been refactored, like the database structure and some dependencies put
    in their rightful place.
    Other important things to this project was to bring an intensive test coverage to the Rails
    API and provide an enjoyable user interface to the clients (NEHiLP researchers)

    \noindent \textbf{Keywords:} dictionary, etymology, database, web development, tests, UI



    % ---------------------------------------------------------------------------- %
    % Sumário
    \dominitoc % inicialização
    \tableofcontents    % imprime o sumário



    %% % ---------------------------------------------------------------------------- %
    %% % Listas de figuras e tabelas criadas automaticamente
    \dominilof \listoffigures \mtcaddchapter
    %% \listoftables



    % ---------------------------------------------------------------------------- %
    % Capítulos do trabalho
    \mainmatter

    % cabeçalho para as páginas de todos os capítulos
    \fancyhead[RE,LO]{\thesection}

    \singlespacing              % espaçamento simples
    %\onehalfspacing            % espaçamento um e meio


    \chapter{Da Introdução} \label{cap:introducao}
\minitoc \mtcskip

\intro{Segundo o \citetalias[p.15]{MN2}, o idioma português é aquele com menor quantidade
de informação etimológica disponível se comparado com outros idiomas românicos. Com
vistas a minorar esse problema, alguns projetos de pesquisa buscam métodos de otimizar a
obtenção de registros. O projeto DELPo é um ótimo exemplo. Através de um algoritmo
\textit{moedor} que recebe um arquivo, registram-se ocorrências, lemas e flexões que
auxiliam a monitorar que elementos são mais recentes. Este fluxo mimetiza o algoritmo que
se aplica a cartões físicos.

O projeto já conta com um portal web em produção. Este trabalho, entretanto, era feito em PHP,
e havia alguns gargalos na aplicação. Alguns foram removidos ano passado, outros estão sendo
extirpados agora. Além disso, cada parte da implementação agora está sendo feita com as ferramentas
que o aluno julga mais apropriadas. Por exemplo, a parte do projeto implementada em Rails (Ruby on
Rails) é o \say{backend} enquanto as visualizações são feitas em \citetalias{Nuxt}}

\section{Uma observação Importante} \label{sec:importante}

Por se tratar da continuação do trabalho de \citet*{delpo:18}, muitas coisas foram
extraídas do trabalho deles. Muitas citações e paráfrases estarão presentes neste
trabalho, principalmente quando se trata dos conceitos relacionados à parte do
cliente. Isso se deve pelo fato óbvio do trabalho ser quase o mesmo, com poucas
diferenças (a proposta é muito semelhante, mas o cliente e a aplicação são os mesmos).
Desta forma, não há motivo razoável para que não se mantenha muito do que foi feito.

\section{Objetivos Específicos}
\label{sec:goals}

Aqui tratar-se-á de todos os objetivos de forma específica e como pretendo cumpri-los.

\textbf{Gerar um sistema eficiente}. Trata-se de uma parte muito importante deste trabalho.
Dado que no sistema legado (em PHP) observavam-se alguns gargalos de desempenho, e que estes
comprometiam a utilização por parte de pesquisadores, este trabalho propõe-se a eliminar, ou
reduzí-los de maneira a facilitar a vida dos entusiastas e estudiosos da área.

Além da motivação nobre de ajudar, não se pode ignorar o impacto de um grande trabalho como
este terá no curriculum do aluno em questão. Um trabalho bem feito abrirá portas para grandes
oportunidades que aparecerão dado que quem faz um excelente serviço poderá fazer muitos outros.
Do contrário, um serviço mal feito, um software de qualidade duvidosa não vai abrir portas ao
aluno, que nem se daria ao trabalho de mostrar algo mal-feito.

\textbf{Verificar a importância real de um código limpo.} Ao início da graduação, o aluno incauto
pode fazer uma vaga ideia, mas jamais entenderá a importância de fazer um código legível
\footnote{Salvo o caso desse aluno já tiver alguma experiência com programação}. Isso muda quando
o aluno têm contato com um código ilegível. A dificuldade em refatorar o código, entender o que se
deseja, ou até mesmo a necessidade de refazer grandes partes daquele programa será traumática o
bastante para que o aluno passe a se preocupar com o que ele faz.

Mas o que é um código de limpo? Segundo \citeauthor{CCL}, é um código que pode ser lido, compreendido,
melhorado, estendido e mantido por qualquer outro desenvolvedor.

Ele fez um trabalho importantíssimo ao juntar vários preceitos em um só documento. Um grande punhado
de convenções (ou conselhos, se preferir) para quem deseja fazer um código excelente. Por se tratar de
mais de 50 \say{leis}, é um tanto trabalhoso que sigam-se todas elas, mas com certeza se trata de um bom
ponto de partida.

Na sessão \ref{sec:boas-praticas} encontram-se algumas diretrizes da produção de bom código.

\section{Sobre a proposta} \label{sec:proposta}

Vale saber, além dos objetivos supra citados, qual é a proposta geral desse trabalho.
\begin{itemize}
    \item continuação do trabalho de criar uma aplicação em Rails\footnote{vide p.\pageref{subsec:rails}}
    da API usada no novo portal do DELPO. Inclui cobertura extensiva de testes de unidade.
    \item retoque no banco de dados da aplicação e articulação das entidades usando PostgreSQL
    \item modificação do modelo de autenticação de usuários e implementação de níveis de permissão compatível
    com o que o \citetalias{MN2} determina
    \item melhoramento do sistema moedor \footnote{vide p.\pageref{subsec:moagem}}
    \item Permitir que a plataforma esteja disponível para uso da melhor forma possível. Ainda que isso
    implique que algumas refatorações e melhorias ficarem de lado.
\end{itemize}

O moedor que há na aplicação atual é enorme e exige grande refatoração. Entretanto, como o tempo disponível não 
era o bastante para sanar os problemas observados, o que será feito são alguma melhorias em alguns métodos,
remoção dos desnecessários, entre outras medidas. A ideia é que, na medida do possível, as funções fiquem mais
fáceis de entender, e o código ganhe em inteligibilidade. O moedor tem dois módulos: o de leitura e o de análise.

O primeiro trata de quebrar o texto em contextos (orações) e palavras. Cada quais destas é analisada com o
objetivo de encontrar sua forma básica\footnote{por exemplo, abrindo \dir abrir}, procurar se ela existe no
banco, bem como datar e retrodatar de acordo com o que for necessário. Os resultados obtidos são armazenados
em uma tabela de mesmo nome.

O segundo começa a partir do ponto em que a leitura termina. Este módulo depende da interação do pesquisador
que faz as partes da tarefas que a máquina não é capaz de fazer. Este módulo toma os resultados obtidos no
módulo acima, e os apresenta ao pesquisador da forma como será mencionada na sub-seção \ref{subsec:nehilp-delpo}
na descrição do programa moedor. Resumidamente, uma palavra contida na obra pode ser marcada em azul (se ainda
não estiver no banco de dados) ou em vermelho (caso já exista no banco de dados e esteja associada a uma data
\emph{maior}\footnote{vide sub-seção \ref{subsec:moagem} para mais detalhes} que a do texto analisado)

\section{Metodologia}
\label{sec:methodology}

A metodologia seguida nesse trabalho consiste de alguns passos:
\begin{enumerate}
    \item Conhecer o projeto a fundo.
    \item Compreender o código.
    \item Entender os relacionamentos entre as entidades do banco de dados
    \item Realizar as mudanças necessárias (enquanto faz os testes)
    \item Fazer a interface visual do sistema
    \item Se possível, criar uma documentação suficiente para a compreensão e
    utilização dessa ferramenta
\end{enumerate}

O passo 1 é o mais básico de todos e consistiu de uma conversa inicial com o professor Mário, atual vice-coordenador
e criador do projeto DELPo. O professor Marco também me ajudou a entender alguns aspectos importantes do assunto.
Entretanto, o \citetalias{MN2} e o artigo de \cite{Mar:17} foram de grande ajuda para entender a diferença entre os
lemas, e tornaram-se poderosas fontes de consulta.

O passo seguinte foi o maior desafio, dado que eu só tenho contato regular com um dos membros
do antigo grupo de trabalho. Isso tornou-se uma dificuldade tamanha e me gerou um grande bloqueio
que me levou a reorganizar as entidades do banco de dados. Ainda que seja notável que a equipe
anterior tem um bom domínio de ferramentas SQL, percebe-se que não se utilizaram plenamente da
interface provida pelo Rails para poder implementar melhor os modelos. Além disso, o algoritmo
do moedor ainda é um pesadelo para quem se aventura a refatorá-lo.

O passo 3 também exigiu um bom tempo. Inicialmente tentou-se fazer menores alterações e manter
o código. Isso seria muito menos trabalhoso, e demandaria menos tempo também. O que se observou,
entretanto é que fazer mudanças específicas não era tão simples quanto parecia, ou quanto de fato
seria se estivesse com um dos membros da antiga equipe consigo. Dessa forma, decidiu-se refazer as
tabelas com base nas que já tinham sido feitas. Enquanto os modelos ficavam prontos e se conectavam,
os testes cobriam se as determinações e os relacionamentos foram realmente estabelecidos.

Além das mudanças supracitadas, o passo 4 engloba as mudanças nos controladores e a criação de métodos
auxiliares (\emph{helpers}). Esses métodos, inseridos em módulos independentes, garantem um código bem
compartimentado, conciso e não repetitivo.

O maior desafio foi refatorar o moedor. Os principais problemas com o código são nomes de métodos que
não descrevem o que ele faz, métodos desnecessários (como o que declara um hash que poderia ser uma
constante dentro do módulo), e a função principal do módulo com mais de mil linhas e com muitas
atribuições desconhecidas.

A parte 5 está em desenvolvimento neste exato momento e conta com o auxílio da professora Vanessa,
também integrante do NEHiLP. Nos capítulos que se seguem, falar-se-á sobre as ferramentas usadas.
Aqui basta dizer que o objetivo principal é proporcionar uma boa experiência ao usuário.

\section{Descrição dos capítulos}
\label{sec:description}

O capítulo 2, \textbf{Dos conceitos principais} funciona como um cabeçalho de um programa. Isso pois, nessa
parte do código é onde ficam os comandos de importação de bibliotecas.

Analogamente, no próximo capítulo, tratar-se-ão de conceitos relativos ao material a ser implementado, e do
jargão a ser utilizado. Detalhes a serem esclarecidos para que o cérebro do leitor "compile" (em outras palavras,
que ele compreenda) o conteúdo das páginas que se seguirão.

No capítulo 3, \textbf{Sobre o desenvolvimento}, tratar-se-ão de detalhes técnicos sobre a implementação. Será uma
descrição a dois níveis, sendo a primeira, uma descrição verbal, e o seguinte, a nível de implementação.

No capítulo 4, \textbf{Dos resultados obtidos}, encontra-se uma reflexão sobre o que se passou além do código, o que
está pronto e o que pode ser feito no futuro.

O último capítulo, \textbf{Das conclusões finais}, faz-se um resumo do que está documentado, destacando o que for mais
conveniente.

    \input cap-conceitos
    \input cap-proposta
    \input cap-resultados
    \input cap-conclusoes
    % grammar girl -- podcast conclusao

    % cabeçalho para os apêndices
    \renewcommand{\chaptermark}[1]{\markboth{\MakeUppercase{\appendixname\ \thechapter}} {\MakeUppercase{#1}}}
    \fancyhead[RE,LO]{}
    \appendix

%    \chapter{Códigos}
\label{cap:code}

\section{Arquivos}
\label{sec:files}

\section{Testes}
\label{sec:tests}

Texto texto texto texto texto texto texto texto texto texto texto texto texto
texto texto texto texto texto texto texto texto texto texto texto texto texto
texto texto texto texto texto texto.
      % associado ao arquivo: 'ape-conjuntos.tex'


    % ---------------------------------------------------------------------------- %
    % Bibliografia
    \backmatter \singlespacing   % espaçamento simples
    \bibliographystyle{plainnat-ime} % citação bibliográfica textual
    \bibliography{bibliografia}  % associado ao arquivo: 'bibliografia.bib'


    %%%  ---------------------------------------------------------------------------- %
    %% % Índice remissivo
    %% \index{TBP|see{periodicidade região codificante}}
    %% \index{DSP|see{processamento digital de sinais}}
    %% \index{STFT|see{transformada de Fourier de tempo reduzido}}
    %% \index{DFT|see{transformada discreta de Fourier}}
    %% \index{Fourier!transformada|see{transformada de Fourier}}

    %% \printindex   % imprime o índice remissivo no documento

\end{document}
