\chapter{Dos Resultados Obtidos}
\label{cap:results}
\minitoc

\intro{Neste capítulo serão mostrados os resultados deste trabalho. Mas como fazer isso se capturas de tela
não são capazes de mostrar a aplicação funcionando ao vivo? A ideia é ilustrar as funcionalidades principais
enquanto o link da aplicação em produção ficará diponível aqui.

Outro ponto importante a mostrar aqui é a cobertura de testes. A cobertura do frot não será tão extensiva
quanto a do back, mas ao comparar com o que se tinha no começo do ano, é infinitamente maior. }

\section{O que foi feito?}
\label{sec:to-be-done}

\subsection{API Rails}\label{subsec:api-rails}

\subsection{Interface}\label{subsec:interface}

\subsection{O moedor}\label{subsec:o-moedor}

Como foi dito na sessão \ref{sec:proposta}, o moedor é um punhado gigantesco de código com uma necessidade enorme de
ser fefatorado. Não porque é ruim ou está incorreto, mas por estar extremamente confuso, com as responsabilidades
misturadas, funções enormes\footnote{Sendo a maior a \texttt{processa\_contextos} com exatas 962 linhas à época da
contagem}. A função \texttt{initialize} também precisam de modificações, mas fica para a list de refatorações a fazer.

A ideia é mover para os \emph{helpers} tudo exceto pela função de inicialização e de leitura. O que for para os helpers,
será testado.

\subsection{Vendo as coisas funcionar}
\label{subsec:ver-funcionando}

Link da aplicação em produção

\subsection{Testes e cobertura}
\label{subsec:testes-cobertura}

\section{O que será feito?}
\label{sec:what-will-be-done}


\subsection{Refatorações}\label{subsec:refatoracoes}

% modeor -- RR
% formularios -- N
% botões -- N
