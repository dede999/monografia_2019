%% ------------------------------------------------------------------------- %%
\chapter{Das Conclusões}
\label{ch:conclusoes}
\minitoc

\intro{Toda a jornada tem pontos de virada e fim. A jornada desta disciplina encerra-se aqui, mas esta
aplicação poderá ser continuada a partir do estado atual. Este capítulo, como já foi dito, resume o que já foi
feito. Postando o olhar sobre o passado e o futuro da aplicação}

\section{A história deste trabalho}
\label{sec:about-it}

Como diz a sabedoria popular, logo após a tempestade vem a calmaria, e este projeto é um bom exemplo disso. Depois de um
ano e um TCC fracassado, em parte por questões de logística inter institucional, e parte por desânimo e procrastinação
minha.

Eu não tenho desejo algum de ser a pessoa que só sabe tratar de sua área. Há muito não tinha. Então, em outubro de 2017
comecei a me organizar para construir, em parceria com outro instituto, uma algoritmo que pudesse lhes interessar. Este
seria meu TCC em 2018.

Houve muitas dificuldades de me reunir com os integrantes desse outro instituto, e quando finalmente esta reunião se fez
possível, recebi um balde de água fria tremendo ao saber que meu tema, tal como o imaginei era inviável.

Depois disso, meu orientador me sugeriu outros temas. Distintos, mas igualmente instigantes. Aí entram em campo os fatores
já citados no primeiro parágrafo, aliados ao compromisso de passar em Análise de Algoritmos, e Laboratório de Métodos
numéricos no mesmo semestre em que tinha que entregar o TCC. Passou o prazo inicial, passou o período de recuperação
\footnote{Na prática, uma extensão do prazo padrão}, e eu não tinha o suficiente para entregar. Chegando o momento de
escolher o tema para o ano de 2019, tinha claro pra mim que faria alguma plataforma útil para o estudo de algum grupo
fora da computação. Uma plataforma de estudo interdisciplinar. E nesse contexto é que o DELPo aparece para mim como o
projeto em curso que mais se encaixava no que eu desejava.

A princípio, devo admitir que passou pela cabeça ir atrás de algum professor que me deu aula na época em que eu cursava
história na FFLCH (Faculdade de Filosofia, Letras e Ciências Humanas), mas o fato de que amigos meus trabalhavam no DELPo,
e por isso mesmo, eu sentia que conhecia bem o projeto. Foi uma decisão sábia, pois por mais trabalho que tivera, é uma
satisfação muito grande poder fazer um trabalho com um impacto potencial tão grande.

Durante o primeiro semestre pouca coisa pude fazer pois peguei demasiadas atividades e matérias, além de um estágio. Este
último foi um divisor de águas em minha opinião. Isso porque tive a oportunidade de usar Rails num ambiente profissional.
Com um grau muito mais alto de cobrança do que as disciplinas em que era necessário fazer uma aplicação. Tive colegas que
me ensinavam coisas sobre a dinâmica daquele ambiente e que sempre me ajudavam a resolver problemas mais complexos. Por
outro lado, tive colegas que mesmo remotamente me cobravam sempre um código o melhor possível, e sempre postavam algum
comentário nos \emph{pull requests} que eu mandava me exigindo mais. Ainda que isso causasse um pequeno aborrecimento,
dado que queria muito participar dos projetos internos e ser designado para algum cliente após se chegasse a ser efetivado,
por outro lado, foi importantíssimo para que eu tivesse contato com melhores práticas de programação. Sou infinitamente
grato pelas ajudas que recebi, e até pelos puxões de orelha, cujo valor eu reconheço.

Não demorou até que o excesso de atividades cobrasse a conta.
No último mês do semestre, fui dispensado do meu estágio.
Foi um duro golpe, é verdade, mas mantive uma boa atitude com a situação.
Com isso, foi fácil notar o quão providencial, ainda que triste, fora sair do estágio àquela altura.
Sem sombra de dúvida eu tinha muitíssimo a aprender ali, mas tinha outras tarefas a concluir também.
Nunca saberei se conseguiria abraçar minhas matérias, a monitoria que eu dava para o professor Gubitoso, o estágio
e o TCC. O que sei é que tudo acabou se ajeitando providencialmente com uma boa dose de esforço pessoal.

A monitoria dada, e o estágio feito são fatores sem os quais o resultado deste trabalho não seria tão bom quanto fora.

No segundo semestre as coisas pesaram, mas nem de perto tanto quanto no primeiro. Por esse e outros motivos é que o
desenvolvimento correu muito bem. Dificuldades ocorreram, é verdade, mas eram tão somente ligadas à aplicação.

O motivo de acreditar que este projeto fora a calmaria pós tempestade, mesmo depois das dificuldades mencionadas, é o fato
de que tudo o que passei me levar a um grande estado de realização pessoal. Coisa que não aconteceu ano passado. Todas as
dificuldades que passei, direta ou indiretamente ligadas ao DELPo, constituem-se em riquezas que levarei pra vida.
Poucas alegrias igualam-se a esta.

\section{O que aprendi com esse trabalho}
\label{sec:lessions}

Eu aprendi muito com este trabalho. Muitas coisas poderiam ser elencadas aqui, mas por simplicidade, cito apenas duas que
são os aprendizados principais.

O primeiro foi durante meu tempo de estágio onde era cobrado sempre a fazer um código excelente. Sempre o melhor possível.
Se não fosse por isso, meu trabalho não ficaria tão bom e eu teria muitos outros problemas com a aplicação Rails. Além do
mais, sem essa postura que adotei pra mim, também não teria o aprendizado que mencionarei a seguir.

Falo do equilíbrio de fazer um código impecável e um código que funcionará e entregará as funcionalidades no prazo
estipulado. Eu não tive este equilíbrio. Por vezes perdi um tempo precioso a pensar numa forma ótima de fazer determinada
coisa, quando o que eu tinha que fazer, na verdade é fazer um bom código no prazo dado, refatorando o código na primeira
oportunidade. Este, na minha opinião é o aprendizado mais importante, e ainda estou refletindo sobre como aplicá-lo daqui
pra frente.

\section{O que será feito?}
\label{sec:what-will-be-done}

Mesmo depois de concluído a disciplina, pretendo ainda contribuir para o projeto, e isso abre oportunidades de implementar
mais coisas, além de terminar o que foi iniciado, claro.

\subsection{Atividades Em Andamento}\label{subsec:ongoing}

Segue um detalhamento sobre os detalhes que faltam para completar os pontos levantados na tabela \ref{table:activity}

\subsubsection{Fazer uma interface de usuário agradável}

\begin{itemize}
  \item Completo o sistema de login, é necessário inserir a funcionalidade de moagem e estudar como serão mostrados os
  resultados
  \item Falta criar um painel de controle do usuário onde ele possa interagir com as atividades que ele pode usar, e fazer
  suas pesquisas --- isso pode depender da implementação de controladores que antes não eram necessários
\end{itemize}

\subsubsection{Fazer os testes da aplicação Rails}

\begin{itemize}
  \item Depende de continuar refatorando código antigo
  \item Como ficou claro na figura \ref{fig:test-coverage}, os arquivos com menos (abaixo de 90\%) cobertura foram
  \begin{itemize}
    \item \texttt{app/models/moedor.rb}
    \item \texttt{app/helpers/moedor\_helper.rb}
  \end{itemize}
  \item No caso desses aquivos, o problema é a quantidade de métodos ainda sem refatoração.
\end{itemize}

\subsubsection{Refatorar o moedor}

O que falta pra concluir essa tarefa é quebrar dois métodos que dependem de consultas em SQL utilizando o
\emph{ActiveRecord}. Outra coisa importantíssima e potencialmente desafiadora é quebrar o método
\texttt{processa\_contextos} que conta com 962 linhas. O \texttt{initialize}  dessa classe (que pode ser engolida pelo
método de mesmo nome) também deve ser quebrado em alguns outros.

\subsection{Atividades A Fazer}\label{subsec:to-do}

\paragraph{Adaptação do conteúdo do outro banco de dados} Pode não ser o maior dos desafios, mas pode dar
bastante trabalho. Trata-se de colocar os dados do banco de dados da aplicação atualmente em produção, e
passá-los para a aplicação Rails.

\paragraph{Fazer os testes do front} A ideia para testar a aplicação em Nuxt é assistir um curso da plataforma
\href{https://vueschool.io/}{\textbf{Vue School}} voltado a como testar componentes Vue com Jest. O propósito é
usar os componentes criados como material de estudo dessas aulas. Depois de testar o que foi feito até agora, é
possível continuar a implementação do ponto em que ela fora interrompida.

\subsection{Refatorações}\label{subsec:refatoracoes}

Além da refatoração do moedor já mencionada, é possível que se implemente um construtor de formulários e botões. Tudo
para repetir menos código e facilitar os testes.

\section{Comentários finais} \label{sec:botton-line}

Como já fora dito, uma quantia de refatorações foram feitas e todas fizeram do DELPo um sistema melhor e mais 
robusto. Como o sistema não estava em um nível de implementação que permitisse que o cliente validasse o
sistema, não há como apontar uma melhoria clara no seu funcionamento. O que melhorou a olhos vistos foi a
qualidade do código. Diminuiram os \say{cheiros} (problemas) do código, mas faltam mais alguns para que fique
com menos déficit técnico possível.

%Web: \url{www.vision.ime.usp.br/~jmena/stuff/tese-exemplo}}.